\documentclass{article}

\usepackage{amsmath}
\usepackage{amsfonts}
\usepackage{amssymb}
\usepackage{bibref}
\usepackage{authblk}
\title{WORDBOX: A case study in SAT Encodings}
\author{Victor S. Miller}
\affil{IDA Center for Communications Research, Princeton, NJ 08540
USA}
\begin{document}
\maketitle{}
\begin{abstract}
  The great effectiveness of SAT solvers for large, difficult, SAT
  problems is a recent welcome development.  It is well know to
  practitioners in that field that the encoding of a problem can have
  a significant and radical effect on the practical solution time.

  In particular, many combinatorial problems have a straightforward
  ``easy'' encoding which is logically sufficient to specify the
  problem.  However, SAT solvers are very poor at inferring ``global''
  knowledge about the problem.  In fact, with some problems, such as
  the ``Pigeon Hole Principal'', in which global knowledge makes the
  problem trivial, branching SAT solvers must take exponential time.
  Examples of global knowledge are graph connectedness and
  parity. Therefore, in order to find an improved encoding global
  knowledge should be used.

  In this note we consider the encoding a recreational Mathematics
  problem known as \emph{Wordbox}.  One is given a list of words, and
  an $m \times n$ rectangular grid.  The object of the problem is to
  label the grid points with letters so that one can trace out all of
  the words in the list by moving adjacent grid points (using up/down,
  right/left moves).  We show that this problem is an instance of the
  \emph{labeled graph homomorphism} problem, in which we're given two
  finite undirected graphs $G$ and $H$, along with labels $\ell(v)$ for
  every node of $G$.  The object of the problem is to find a map
  $f: V(G) \rightarrow V(H)$ such that if $(v,v') \in E(G)$ then
  $(f(v), f(v')) \in E(H)$ (this is a \emph{graph homomorphism}), and
  a labeling $\ell(w)$ of $w \in V(H)$ such that, for all $v \in V(G)$
  we have $\ell(f(v)) = \ell(v)$.

  We investigate a number of different encodings of the global
  knowledge and their effect on solving times for various SAT solvers.
\end{abstract}

\section{A logical statement}
\label{sec:logical-statement}

We may state the problem in logic as follows
\begin{equation}
  \label{eq:1}
  \exists f \forall v,v' \in V(G) ((v,v') \in E(G) \Rightarrow (f(v),
  f(v')) \in E(H)) \wedge (\ell(v) \ne \ell(v') \Rightarrow f(v) \ne f(v')),
\end{equation}
where $f$ is a function.  However, in order to encode this is
quantifier normal form we need to encode $f$ as a binary relation.

So $w=f(x)$ means
\begin{displaymath}
  \forall x \exists w R(x,w) \wedge \forall x,w,w' ((w = w') \vee
  \neg R(w,x) \vee \neg R(x,w')).
\end{displaymath}
\end{document}
