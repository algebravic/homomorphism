\documentclass{article}

\usepackage{amsmath}
\usepackage{amsfonts}
\usepackage{amssymb}
\usepackage{amsthm}
\usepackage{bibref}
\usepackage{authblk}
\usepackage{hyperref}
\title{WORDBOX: A case study in SAT Encodings}
\author{Victor S. Miller}
\affil{IDA Center for Communications Research, Princeton, NJ 08540
USA}
\newtheorem{definition}{Definition}
\begin{document}
\maketitle
\begin{abstract}
  The great effectiveness of SAT solvers for large, difficult, SAT
  problems is a recent welcome development.  It is well know to
  practitioners in that field that the encoding of a problem can have
  a significant and radical effect on the practical solution time.

  In particular, many combinatorial problems have a straightforward
  ``easy'' encoding which is logically sufficient to specify the
  problem.  However, SAT solvers are very poor at inferring ``global''
  knowledge about the problem.  In fact, with some problems, such as
  the ``Pigeon Hole Principal'', in which global knowledge makes the
  problem trivial, branching SAT solvers must take exponential time.
  Examples of global knowledge are graph connectedness and
  parity. Therefore, in order to find an improved encoding global
  knowledge should be used.

  In this note we consider the encoding a recreational Mathematics
  problem known as \emph{Wordbox}.  One is given a list of words, and
  an $m \times n$ rectangular grid.  The object of the problem is to
  label the grid points with letters so that one can trace out all of
  the words in the list by moving adjacent grid points (using up/down,
  right/left moves).  We show that this problem is an instance of the
  \emph{labeled graph homomorphism} problem, in which we're given two
  finite undirected graphs $G$ and $H$, along with labels $\ell(v)$ for
  every node of $G$.  The object of the problem is to find a map
  $f: V(G) \rightarrow V(H)$ such that if $(v,v') \in E(G)$ then
  $(f(v), f(v')) \in E(H)$ (this is a \emph{graph homomorphism}), and
  a labeling $\ell(w)$ of $w \in V(H)$ such that, for all $v \in V(G)$
  we have $\ell(f(v)) = \ell(v)$.  In turn, we show that the labeled
  graph homomorphism problem is a subset of the \emph{colored graph
    homomorhphism problem}.

  We investigate a number of different encodings of the global
  knowledge and their effect on solving times for various SAT solvers.
\end{abstract}

\section{A recreational problem}
\label{sec:recreational}

In the GCHQ \emph{Kryptos Kristmas Kwiz} of 2006
\url{https://theintercept.com/2015/12/25/gchq-play-a-british-spy-game/}
\footnote{Solutions at \url{https://theintercept.com/2016/01/01/gchq-play-a-british-spy-game-the-solutions/}}
describes a problem called ``wordbox'':
\begin{quotation}
  A wordbox is a rectangle of letters in which a series of words can
  be read by starting at one letter and moving to an adjacent letter
  in a row or column, but not diagonally. Retracing steps and reusing
  letters is allowed.  Thus a wordbox might be
  \begin{center}
    \begin{tabular}[h]{|c|c|c|c|}
      \hline
       D & O & R & F \\
      \hline
       I & G & E & S \\
      \hline
       T & A & O & T \\
      \hline
    \end{tabular}
  \end{center}
  which contains {\tt DODO, DOG, FROG, STOAT TIGER} and {\tt TIT}, but
doesn't contain {\tt GOAT} or {\tt GEESE}. This wordbox contains 12
letter cells, so its area is 12.  What is the smallest (in area)
wordbox that includes the planets: {\tt MERCURY, VENUS, EARTH, MARS,
  JUPITER, URANUS, NEPTUNE, PLUTO}?
\end{quotation}
The answer given (without any proof of minimality) is:
\begin{center}
  \begin{tabular}[h]{|c|c|c|c|}
    \hline
    V & E & A & M \\
    \hline
    A & N & R & T \\
    \hline
    R & U & S & H \\
    \hline
    N & T & A & \quad \\
    \hline
    E & P & I & M \\
    \hline
    L & U & T & E \\
    \hline
    P & J & O & R \\
    \hline
    Y & R & U & C \\
    \hline
  \end{tabular}
\end{center}
\section{A logical statement}
\label{sec:logical-statement}

To solve the original recreational problem, we pose it as a problem of
\emph{Labeled graph homomomorphism}:

\begin{definition}[Graph Homomorphism]
  Let $G, H$ be graphs.  A \emph{graph homomorphism} from $G$ to $H$
  is a map $\phi : V(G) \rightarrow V(H)$ such that, for all $(v,w) \in
  E(G)$, we have $(\phi(v), \phi(w)) \in E(H)$.
\end{definition}
\begin{definition}[Labeled Graph]
  Let $G$ be a graph.  A \emph{labeling} of $G$ is a map $\ell: V(G)
  \rightarrow C$, for some set $C$.
\end{definition}
\begin{definition}[Labeled homomorphism]
  Let $G$ be a labeled graph, with a \emph{labeling} $\ell: V(G)
  \rightarrow C$, and $H$
  be a graph.  A \emph{labeled homomorphism} is a homomorphism $\phi:
  V(G) \rightarrow V(H)$, and a labeling $\ell': V(H) \rightarrow C$,
  such that $\ell'(\phi(v)) = \ell(v)$ for all $v \in V(G)$.
\end{definition}

For the original problem, the graph $G$ is a disjoint union of simple
paths: for each word, of length $k$, there is a path $v_1, \dots,
v_k$, where the edges are $(v_i, v_{i+1})$ for $i=1, \dots, k-1$, and
$\ell(v_i)$ is the $i$-th letter in the word.  The graph $H$ is an $m
\times n$ \emph{grid graph}: its vertices are pairs $(i,j), 0 \le i <
m, 0 < j < n$, with $((i,j), (i',j'))$ an edge if and only if
$|i-i'| + |j-j'| = 1$.

We may state the problem as follows: We are given two
undirected graphs $G, H$, and a \emph{labeling} $\ell : V(G)
\rightarrow C$.  We want to know if there exists a graph homomorphism
$f: V(G) \rightarrow V(H)$, and a labeling $\ell': V(H) \rightarrow C$
such that $\ell'(f(v)) = \ell(v)$ for all $v \in V(G)$.


For the original problem, the graph $G$ is a disjoint union of paths,
$H$ is the \emph{grid graph}
\begin{equation}
  \label{eq:1}
  \begin{aligned}
    \exists & f: V(G) \rightarrow V(H), \ell': V(H) \rightarrow C \\
    \forall & (v,v') \in E(G), (f(v), f(v')) \in E(H) \\
    \wedge \forall & v \in V(G), \ell'(f(v)) = \ell(v).
  \end{aligned}
\end{equation}
where $f$ is a function.  However, in order to encode this in
quantifier normal form we need to encode $f$ as a binary relation.

So $w=f(x)$ means
\begin{displaymath}
  \forall x \exists w R(x,w) \wedge \forall x,w,w' ((w = w') \vee
  \neg R(w,x) \vee \neg R(x,w')).
\end{displaymath}

\section{The colored homomorphism problem}
\label{sec:colored}

\begin{definition}[Edge Colored Graph]
  An \emph{edge colored graph} is a finite undirected graph $G =
  (V(G), E(G))$ a set, $C$ of colors and a map $\phi: E(G) \rightarrow
  C$, which assigns a color to each edge.
\end{definition}
\begin{definition}[Edge Colored Homomorphism]
Let $G$ and $H$ be two edge colored graphs with the same set of
colors.  An \emph{edge colored homomorphism} is a homomorphism $f : G
\rightarrow H$, such that, for all edges $e \in E(G)$, the color of
the edge $f(e) \in E(H)$ is the same as that of $e$.
\end{definition}

We now reduce the vertex colored homomorphism problem to the 2-colored
edge homomorphism problem.  Given a vertex colored graph $G$,
construct the edge colored graph $G'$ with two colors, red and green.
The edges of $G'$ will be a copy of the edges of $G$ colored green.
For each $v \ne v' \in V(G)$ such that $\ell(v) \ne \ell(v')$, and
$(v,v') \not \in E(G)$, we add an edge $(v,v')$ colored red.  For the
target graph $H$, we construct a 2-colored graph $H'$, whose green
edges are a copy of the edges of $H$, and whose red edges are
$(v,v') \not \in H$, for $v \ne v'$.
\begin{proposition}
  There is a vertex colored homomorphism from $G$ to $H$ if and only
  if there is an edged colored homomorphism from $G'$ to $H'$.
\end{proposition}
\end{document}
